\documentclass{beamer}
\usepackage{graphicx}
\logo{\includegraphics[height=0.7cm]{images.png}}

\usepackage[ruled,french,onelanguage]{algorithm2e}

%\usepackage{palatino}
\usetheme{Madrid}
\usecolortheme{whale}
\usepackage{movie15}
\setbeamercolor{frametitle}{fg=white,bg=blue}

\title{Soutenance du projet 2}
\subtitle{Jeu de Tron}
\author{Wassim Rafik Louheb Massinissa}
\institute[Unicaen]
{
  UFR des Sciences\\
  Université de Caen Normandie 
}
\date{04 Avril 2024}

\begin{document}

\begin{frame}
\begin{figure}
        \centering
        \includegraphics[width=0.5\textwidth]{pngegg.png} 
    \end{figure}
  \titlepage
\end{frame}

\begin{frame}
  \frametitle{Sommaire}
  \tableofcontents
\end{frame}

\section{Présentation du Jeu}
\begin{frame}{Présentation du Jeu}
\centering
    \LARGE\textbf{C'est quoi le jeu de tron ?}
    
\end{frame}

\section{Objectifs du Projet}

\begin{frame}{Objectifs du Projet}
    \begin{block}{}
        \begin{itemize}
            \item Implémentation d'un jeu de Tron (sans joueur humain).
            \item Visualisation du jeu.
             \item Implémentation de la méthode de Voronoi. 
            \item Implémentation des algorithmes MAXN, Paranoid et SOS.
            \item Réalisation d'expérimentations.
        \end{itemize}
    \end{block}
\end{frame}



\section*{Modélisation}

\begin{frame}{Modélisation}
    \begin{block}{Parties principales du jeu:}
        \begin{itemize}
            \item Actions possibles : Up, Down, Left, Right.
            \item Représentation d'un joueur.
            \item Représentation de la grille du jeu.
            \item Représentation d'un état du jeu.
            \item Mise en place des Algorithmes.
        \end{itemize}
    \end{block}
\end{frame}



\section{Algorithmes implementés}
\begin{frame}{Algorithme MaxN}
    \begin{block}{Principe et Fonctionnement}
        L'algorithme MaxN explore de manière récursive les actions possibles dans un jeu à plusieurs joueurs, sélectionnant finalement l'action qui maximise le gain pour le joueur Max.
    \end{block}
    \begin{figure}
        \centering
        \includegraphics[width=0.7\textwidth]{Maxn.png} 
        % \caption{Tron - Online Java Game Preview}
    \end{figure}
\end{frame}

\section*{Algorithme Paranoid}

\begin{frame}{Algorithme Paranoid}
    \begin{block}{Principe et Fonctionnement}
      L'algorithme Paranoid repose sur une stratégie paranoïaque, où le joueur principal anticipe la coopération des autres joueurs pour minimiser son score, ainsi il se déplace pour minimiser son propre risque.
    \end{block}
    \begin{figure}
        \centering
        \includegraphics[width=0.7\textwidth]{paranoid.png} 
        % \caption{Tron - Online Java Game Preview}
    \end{figure}
\end{frame}

\section*{Algorithme SOS}

\begin{frame}{Algorithme SOS}
    \begin{block}{Principe et Fonctionnement}
        L'algorithme SOS équilibre stratégiquement les équipes en favorisant à la fois la coopération et la compétition entre les joueurs. En prenant en compte les relations sociales, il vise à maximiser à la fois la survie individuelle et la réussite collective.
    \end{block}
    \begin{figure}
        \centering
        \includegraphics[width=0.6\textwidth]{sos.png}
        % \caption{Tron - Online Java Game Preview}
    \end{figure}
\end{frame}

\section{Evaluation du jeu}

\begin{frame}{Evaluation de base}
    \begin{exampleblock}{Les heuristiques utilisées sont :}
        \begin{itemize}
            \item La distance par rapport aux autres joueurs.
            \item La distance par rapport aux murs.
            \item La distance par rapport au centre de la grille.
            \item Le nombre de cases vides autour du joueur.
            \item La distance par rapport aux bords de la grille.
        \end{itemize}
    \end{exampleblock}
\end{frame}

\begin{frame}{Méthode Voronoi dans l'évaluation du jeu}

\begin{block}{Principe et fonctionnement}
    
    \begin{itemize}
        \item La méthode Voronoi est utilisée dans l'évaluation des jeux pour déterminer les zones de contrôle des joueurs.
    \end{itemize}
\end{block}
    
    \begin{figure}
     \begin{center}
        \includegraphics[width=0.2\textwidth]{voronoi_1.png}\hspace{5mm}
        \includegraphics[width=0.2\textwidth]{voronoi_9.png}\hspace{5mm}
        \includegraphics[width=0.2\textwidth]{voronoi_13.png}
        \caption{Exemple région voronoi }
        \label{fig:voronoi_images}
     \end{center}
\end{figure}

\end{frame}

\begin{frame}{Interface Graphique}

     
   \begin{figure}
        \centering
        \includegraphics[width=0.8\textwidth]{ecran_interface.png} 
        \caption{Une demo de notre jeu de tron}
    \end{figure}
    
\end{frame}
\section{Expérimentations}
\begin{frame}{Résultats des expérimentations}
    \begin{exampleblock}{\textbf{Représentation des résultats}}
        \begin{figure}[htbp]
    \centering
    \begin{minipage}[b]{0.52\textwidth}
        \centering
        \includegraphics[width=\textwidth]{resultats_exp_taille_grille.png}
        \caption{Une partie entre un MaxN, un Paranoid et deux aléatoires en variant la taille de la grille.}
    \end{minipage}
    \hfill
    \begin{minipage}[b]{0.45\textwidth}
        \centering
        \includegraphics[width=\textwidth]{sos_maxn_profondeur_4.png}
        \caption{Une partie entre deux équipes, une SOS contre une MaxN en variant la taille des équipes.}
    \end{minipage}
\end{figure}
    \end{exampleblock}
\end{frame}

\section{Répartition des tâches}

\begin{frame}{Répartition des tâches principales}
    \begin{block}{Tâches attribuées aux membres de l'équipe :}
        \begin{itemize}
            \item \textbf{ Wassim :}
                \begin{itemize}
                     Implémentation de l'algorithme MaxN et Paranoid 
                \end{itemize}
            \item \textbf{ Rafik :} 
                \begin{itemize}
                     Implémentation de la méthode de Voronoi pour l'évaluation
                \end{itemize}
            \item \textbf{ Louheb :} 
                \begin{itemize}
                    Implémentation de l'algorithme  SOS 
                \end{itemize}
            \item \textbf{ Massinissa :} 
                \begin{itemize}
                    Implémentation de l'interface graphique
                \end{itemize}
        \end{itemize}
    \end{block}
\end{frame}


\section{Améliorations}

\begin{frame}{Améliorations}
    \begin{center}
        \LARGE Améliorations 
    \end{center}
\end{frame}
\section{Conclusion}

\begin{frame}{Conclusion}
    \begin{center}
        \LARGE Conclusion 
    \end{center}
\end{frame}

\end{document}