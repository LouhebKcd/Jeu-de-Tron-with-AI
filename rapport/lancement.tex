\section{Comment lancer le projet?}
\subsection{Lancement du projet}

Pour lancer le projet, suivez ces étapes :

\begin{enumerate}
  \item \textbf{Compilation des classes :} Pour compiler toutes les classes, exécutez la commande suivante à partir du répertoire racine du projet :
 
  \begin{verbatim}
  javac -d build src/model/algorithmes/*.java src/model/evaluation/*.java
  src/model/jeu/*.java src/model/main/*.java src/vue/*.java
  \end{verbatim}
 
  \item \textbf{Exécution des classes exécutables :} Une fois que toutes les classes ont été compilées, vous pouvez exécuter chaque classe exécutable individuellement :
 
  \begin{itemize}
    \item \textbf{Exécution de la classe principale DemoParametrable :}
    
    \begin{verbatim}
    java -cp build model.main.DemoParametrable
    \end{verbatim}
    
   \item \textbf{Exécution de la classe principale Main :} Pour exécuter la classe principale `Main` en spécifiant les arguments nécessaires, utilisez la commande suivante :
 
  \begin{verbatim}
    java -cp build model.main.Main <taille_grille> <profondeur_recherche>
    <nombre_joueurs>
    \end{verbatim}

    
   \item \textbf{Exécution de la classe principale MainSos :} Pour exécuter la classe principale \texttt{MainSos} en spécifiant les arguments nécessaires, utilisez la commande suivante :
    
    \begin{verbatim}
    java -cp build model.main.MainSos <taille_grille> <profondeur_recherche>
    <nombre_joueurs_par équipe>
    \end{verbatim}
    
    \item \textbf{Exécution de la classe principale MainInter :} Pour lancer l'interface graphique du jeu, exécutez la classe principale \texttt{MainInter}. Utilisez la commande suivante :
   
    \begin{verbatim}
    java -cp build vue.MainInter
    \end{verbatim}
  \end{itemize}
 
  \item \textbf{Utilisation des fichiers JAR :} Vous trouverez les fichiers JAR dans le dossier \texttt{dist}. Une fois que vous avez les fichiers JAR, vous pouvez lancer le projet en utilisant les commandes suivante :
    
    \begin{verbatim}
    java -jar JeuTronConsole.jar
    \end{verbatim}
    
    Cette commande exécutera le jeu sur le terminal en utilisant le fichier JAR pour la version console du jeu.

    \begin{verbatim}
    java -jar JeuTronVueGraphique.jar
    \end{verbatim}
    
    Cette commande exécutera une partie du jeu avec une interface graphique en utilisant le fichier JAR pour la version avec visualisation graphique du jeu.
\end{enumerate}
