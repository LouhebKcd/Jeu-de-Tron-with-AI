\section{Conclusion}

\subsection{Récapitulatif de la problématique et de la réalisation}

Dans ce projet, nous avons abordé le défi de développer des algorithmes d'intelligence artificielle pour le jeu de Tron multi-joueur et coalitions. Notre objectif principal était de mettre en œuvre des stratégies efficaces basées sur des méthodes telles que MAXN, Paranoid et SOS, tout en explorant l'utilisation de la méthode  Voronoi pour améliorer l'évaluation du jeu en plus des heuristiques de base. nous avons accordé une attention particulière aux expérimentations, les considérant comme un pilier essentiel pour évaluer et comparer les performances de nos joueurs solitaires et des coalitions. Ces expérimentations ont été menées dans diverses configurations de jeu, couvrant une gamme étendue de paramètres tels que la profondeur de recherche, la taille de la grille et le nombre de joueurs. Ces tests approfondis nous ont permis de recueillir des données précieuses sur la manière dont nos algorithmes se comportent dans des conditions variées.

\subsection{Propositions d’améliorations}

Bien que notre projet ait atteint ses objectifs initiaux, plusieurs axes d'amélioration peuvent être envisagés pour renforcer l'ensemble du projet et augmenter son efficacité globale :

\vspace{10pt}
\begin{itemize}
    \item Nous pourrions explorer des variantes plus sophistiquées des algorithmes existants, telles que des améliorations de Maxn prenant en compte des facteurs de coupure alpha-bêta.
    \item L'amélioration de l'interface utilisateur pourrait offrir une meilleure expérience de jeu, avec des options plus riches et un design visuel plus attrayant. 
    \item Nous pourrions également envisager d'explorer des techniques d'évaluation plus sophistiquées ou novatrices afin d'améliorer la précision et l'efficacité de l'évaluation des performances des joueurs dans le jeu.
    \item L'intégration de techniques d'apprentissage automatique pour l'adaptation dynamique des stratégies en fonction du comportement des adversaires, représentent des pistes prometteuses pour améliorer l'efficacité de nos joueurs dans des environnements de jeu complexes et changeants.

\end{itemize}