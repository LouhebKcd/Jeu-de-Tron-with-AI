\section{Introduction}
\subsection{Description générale du projet }

Le Jeu de Tron multi-joueur et coalitions s'inscrit dans le cadre de l'exploration des stratégies avancées dans le domaine des jeux à plusieurs joueurs,ce dernier est une variante multi-joueurs du classique jeu du serpent.
Le but du projet comprendre comment les robots peuvent jouer intelligemment dans ce genre de jeu où ils doivent rivaliser et parfois coopérer pour gagner. On étudie  les interactions stratégiques et les performances des algorithmes d'intelligence artificielle  dans un contexte compétitif.

Dans ce jeu, chaque joueur contrôle un point mobile sur une grille fixe et doit naviguer pour éviter les collisions avec les murs, les adversaires ou les bords du plateau. Chaque déplacement laisse derrière lui un mur infranchissable, et l'objectif ultime est de survivre plus longtemps que les autres joueurs. 

Le projet vise à étendre les possibilités du jeu en introduisant des aspects multi-joueurs plus complexes. Pour ce faire, nous allons implémenter des algorithmes tels que MAXN et Paranoid et aussi SOS pour la gestion des équipes, et explorer des stratégies de coalition où plusieurs joueurs s'unissent pour atteindre un objectif commun. 

Une partie importante de ce projet consiste à évaluer l'impact de différentes profondeurs de recherche sur les performances des joueurs solitaires et des coalitions, en tenant compte de facteurs tels que la taille de la grille et le nombre de joueurs dans chaque équipe. Cette analyse permettra de mieux comprendre les mécanismes sous-jacents aux interactions entre les joueurs et d'identifier les stratégies les plus efficaces dans différentes configurations de jeu.

En résumé, ce projet offre une occasion unique d'explorer les complexités des jeux multi-joueurs et de développer des algorithmes pour relever les défis posés par ces environnements compétitifs.
